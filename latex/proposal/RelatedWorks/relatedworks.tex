\section{Related works}
As aforementioned, Text Classification methods can be broadly divided into two main branches: shallow learning approaches and deep learning approaches.

Shallow learning approaches are traditional machine learning techniques that involve the extraction of hand-crafted features from text, such as bag-of-words or n-grams. These methods include Support Vector Machines (SVM)\cite{cortesvapnik1995, boser1992}, Naive Bayes\cite{Xu2017}, Decision Trees\cite{Safavian1991}, and Logistic Regression\cite{Genkin2007, Krishnapuram2005}. While these methods are computationally efficient, they require extensive feature engineering and may not perform well with complex data.

Deep learning approaches, on the other hand, learn features automatically from the data, eliminating the need for extensive feature engineering. These methods include Convolutional Neural Networks (CNN)\cite{Kim2014}, Recurrent Neural Networks (RNN)\cite{Cho2014, Sutskever2014}, combinations between CNN and LSTM\cite{Vo2017}, and Transformer-based models (BERT\cite{DevlinCLT19}, XLNet\cite{Yang2019}). These models have shown state-of-the-art performance on many text classification tasks but require significantly more computational resources and data to train effectively. However, these models are highly scalable, making them suitable for processing large amounts of text data.